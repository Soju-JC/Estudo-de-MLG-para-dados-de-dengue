\documentclass[12pt,a4paper]{article}\usepackage[]{graphicx}\usepackage[]{color}
% maxwidth is the original width if it is less than linewidth
% otherwise use linewidth (to make sure the graphics do not exceed the margin)
\makeatletter
\def\maxwidth{ %
  \ifdim\Gin@nat@width>\linewidth
    \linewidth
  \else
    \Gin@nat@width
  \fi
}
\makeatother

\definecolor{fgcolor}{rgb}{0.345, 0.345, 0.345}
\newcommand{\hlnum}[1]{\textcolor[rgb]{0.686,0.059,0.569}{#1}}%
\newcommand{\hlstr}[1]{\textcolor[rgb]{0.192,0.494,0.8}{#1}}%
\newcommand{\hlcom}[1]{\textcolor[rgb]{0.678,0.584,0.686}{\textit{#1}}}%
\newcommand{\hlopt}[1]{\textcolor[rgb]{0,0,0}{#1}}%
\newcommand{\hlstd}[1]{\textcolor[rgb]{0.345,0.345,0.345}{#1}}%
\newcommand{\hlkwa}[1]{\textcolor[rgb]{0.161,0.373,0.58}{\textbf{#1}}}%
\newcommand{\hlkwb}[1]{\textcolor[rgb]{0.69,0.353,0.396}{#1}}%
\newcommand{\hlkwc}[1]{\textcolor[rgb]{0.333,0.667,0.333}{#1}}%
\newcommand{\hlkwd}[1]{\textcolor[rgb]{0.737,0.353,0.396}{\textbf{#1}}}%
\let\hlipl\hlkwb

\usepackage{framed}
\makeatletter
\newenvironment{kframe}{%
 \def\at@end@of@kframe{}%
 \ifinner\ifhmode%
  \def\at@end@of@kframe{\end{minipage}}%
  \begin{minipage}{\columnwidth}%
 \fi\fi%
 \def\FrameCommand##1{\hskip\@totalleftmargin \hskip-\fboxsep
 \colorbox{shadecolor}{##1}\hskip-\fboxsep
     % There is no \\@totalrightmargin, so:
     \hskip-\linewidth \hskip-\@totalleftmargin \hskip\columnwidth}%
 \MakeFramed {\advance\hsize-\width
   \@totalleftmargin\z@ \linewidth\hsize
   \@setminipage}}%
 {\par\unskip\endMakeFramed%
 \at@end@of@kframe}
\makeatother

\definecolor{shadecolor}{rgb}{.97, .97, .97}
\definecolor{messagecolor}{rgb}{0, 0, 0}
\definecolor{warningcolor}{rgb}{1, 0, 1}
\definecolor{errorcolor}{rgb}{1, 0, 0}
\newenvironment{knitrout}{}{} % an empty environment to be redefined in TeX

\usepackage{alltt}
\usepackage[utf8]{inputenc}
\usepackage{amsmath}
\usepackage{amsfonts}
\usepackage{amssymb}
\usepackage{makeidx}
\usepackage{graphicx}
\usepackage{booktabs}
\usepackage{xcolor}
\let\lctau\tau % save the lowercase of '\tau'
\renewcommand{\tau}{\scalerel*{\lctau}{X}}


\title{MLG - Trabalho 2}
\author{José Carlos Soares Junior }
\date{Matrícula: 2017100732}
\IfFileExists{upquote.sty}{\usepackage{upquote}}{}
\begin{document}
\begin{titlepage}
	\begin{center}
	
	%\begin{figure}[!ht]
	%\centering
	%\includegraphics[width=2cm]{c:/ufba.jpg}
	%\end{figure}

		\Huge{UNIVERSIDADE FEDERAL DO ESPÍRITO SANTO}\\
		\large{CENTRO DE CIÊNCIAS EXATAS}\\ 
		\large{DEPARTAMENTO DE ESTATÍSTICA}\\ 
\vspace{15pt}
        
        \vspace{85pt}
        
		\textbf{\LARGE{Modelagem dos dados de 2013 referentes às notificações de dengue no estado do Espírito Santo}}
		\title{\large{Título}}
	%	\large{Modelo\\
     %   		Validação do modelo clássico}
			
	\end{center}
\vspace{1,5cm}
	
	\begin{flushright}

   \begin{list}{}{
      \setlength{\leftmargin}{4.5cm}
      \setlength{\rightmargin}{0cm}
      \setlength{\labelwidth}{0pt}
      \setlength{\labelsep}{\leftmargin}}

      \item Segundo trabalho da disciplina de MLG ministrado pelo Prof. Dr. Saulo Morellato.

      \begin{list}{}{
      \setlength{\leftmargin}{0cm}
      \setlength{\rightmargin}{0cm}
      \setlength{\labelwidth}{0pt}
      \setlength{\labelsep}{\leftmargin}}

			\item Alunos: \
            \item Orientador: Prof. Dr. Saulo Morellato\

      \end{list}
   \end{list}
\end{flushright}
\vspace{1cm}
\begin{center}
		\vspace{\fill}
		 Abril\\
		 2021
			\end{center}
\end{titlepage}
\newpage

% Pacotes:



\tableofcontents  % sumário

% Formato do chunk:



\newpage
\section{{\LARGE\textbf{Análise exploratória}}}


\subsection{\textbf{Descrição dos dados}}


\newpage
\section{{\LARGE\textbf{Construção do modelo}}}

\subsection{\textbf{Definição do modelo}}

\subsection{\textbf{Modelo considerando todas as covariáveis}}

\subsection{\textbf{Verificando multicolinearidade}}




\subsection{\textbf{Seleção de covariáveis}}


\subsection{\textbf{Transformação de covariáveis}}

\newpage
\section{\textbf{Pontos influentes/de alavanca/outliers}}

\newpage
\section{{\LARGE\textbf{Verificação dos pressupostos}}}

\newpage
\section{{\LARGE\textbf{Interpretação e conclusões}}}

\end{document}
